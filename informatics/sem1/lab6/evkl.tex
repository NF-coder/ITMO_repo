\documentclass[a4paper]{article}

\usepackage[utf8x]{inputenc} % поддержка utf8
\usepackage[english,russian]{babel} % переносы
\usepackage{multicol}
\usepackage{adjustbox}
\usepackage{amsmath}
\usepackage{tikz} % рисование
\usepackage{amssymb}
\usepackage{tabularx} % таблички
\usepackage{setspace}

\usepackage[margin=2cm,bottom=1.5cm]{geometry} % отступ сбоку

\usepackage{graphicx}
\usepackage{wrapfig} % удержание текста около фигур

\pagestyle{empty} % убираем нумерацию

\definecolor{myorange}{HTML}{d94d1a}
\definecolor{myyellow}{HTML}{f2b31a}

\newcommand{\solidline}[6]{
    \begin{tikzpicture}[scale=1.5, line width=2pt]
        \draw[#6, #3] (#4,1) -- (#5,1);
        
        \node[above] at (#4,1) {\tiny #1};
        \node[above] at (#5,1) {\tiny #2};
    \end{tikzpicture}
}

\newcommand{\longline}[3]{
    \solidline{#1}{#2}{#3}{1}{1.6}{}
}

\newcommand{\shortline}[3]{
    \solidline{#1}{#2}{#3}{1}{1.4}{}
}

\newcommand{\shortdashedline}[3]{
    \solidline{#1}{#2}{#3}{1}{1.4}{dashed}
}

\newcommand{\splitline}[2]{
    \begin{tikzpicture}[scale=1.5, line width=2pt]
        \draw[black] (1,1) -- (1.3,1);
        \draw[dashed, black] (1.3,1) -- (1.6,1);
        \node[above] at (1,1) {\tiny #1};
        \node[above] at (1.6,1) {\tiny #2};
    \end{tikzpicture}
}



\begin{document}


\begin{tabular}{p{0.2\textwidth}p{0.7\textwidth}}
    test

    &

    {4\raisebox{0.1em}{0}} \hspace{1cm} КНИГА I ПРЕДЛ. XVI. ТЕОРЕМА

    \begin{wrapfigure}[7]{l}{0.19\textwidth}
        \includegraphics[width=0.2\textwidth]{capital.png}
    \end{wrapfigure}
    \begin{spacing}{1.2}
        \large \textit{ \newline ри продолжении стороны треугольника}
        \begin{tikzpicture}[scale=0.5, line width=2.5pt]
            \node[below left] at (0,1) {\small A};
            \node[above] at (1,4) {\small B};
            \node[below right] at (2,1) {\small C};
            \draw[dashed, myyellow] (1, 4) -- (0,1);
            \draw[dashed, blue] (1, 4) -- (1.5, 2.5);
            \draw[blue] (1.5, 2.5) -- (2, 1);
            \draw[black] (0,1) -- (2, 1);
        \end{tikzpicture}
        \large \textit{ внешний угол }
    \end{spacing}
    

\end{tabular}



\end{document}