\documentclass[a4paper]{article}

\usepackage[T1,T2A]{fontenc}
\usepackage[utf8x]{inputenc} % поддержка utf8
\usepackage[english,russian]{babel} % переносы
\usepackage{multicol}
\usepackage{adjustbox}
\usepackage{amsmath}
\usepackage{tikz} % рисование
\usepackage{amssymb}
\usepackage{tabularx} % таблички
\usepackage{setspace}

\usepackage[margin=2cm,bottom=1.5cm]{geometry} % отступ сбоку

\usepackage{graphicx}
\usepackage{wrapfig} % удержание текста около фигур

\pagestyle{empty} % убираем нумерацию

\definecolor{myorange}{HTML}{d94d1a}
\definecolor{myblue}{HTML}{265999}
\definecolor{myyellow}{HTML}{f2b31a}


\geometry{top=4em,right=2em,left=2em,bottom=4em}

\newcommand{\myheading}{
    \begin{flushleft}
        \raisebox{0pt}[\headheight][30pt]{
            \vbox{
                {4\raisebox{0.1em}{0}} \hspace{1cm} КНИГА I ПРЕДЛ. XVI. ТЕОРЕМА
            }   
        }
    \end{flushleft}
}

\begin{document}

\begin{minipage}[t]{0.3\textwidth}
    \hfill \\
    \begin{center}
        \begin{tikzpicture}
            \draw[myblue,fill=myblue] (-2, -4) --  (-1.4,-4) arc(0:55:0.7) -- cycle;
            \draw[black,fill=black] (0, 0) -- (0.35, -0.65)  arc(-60:-115:0.7) -- cycle;
            \draw[myyellow,fill=myyellow] (1, -2) -- (1.3, -2.6)  arc(-50:35:0.7) -- cycle;
            \draw[myyellow,fill=myyellow] (1, -2) -- (0.5, -2.3)  arc(-160:-235:0.7) -- cycle;
            \draw[myorange,fill=myorange] (2, -4) -- (1.4, -4)  arc(185:290:0.7) -- cycle;
            \draw[black, ultra thick,fill=black] (2, -4) -- (1.7, -3.4)  arc(-250: -295: 0.7) -- cycle;
            \draw[black,ultra thick] (2, -4) -- (2.5, -4)  arc(-4:50:0.7) -- cycle;
    
            \draw[dashed, myyellow, ultra thick] (-2,-4) --  (0,0);
            \draw[myblue, ultra thick] (0,0) --  (1, -2);
            \draw[dashed, myblue, ultra thick] (1,-2) --  (2, -4);
            \draw[black, ultra thick] (-2,-4) --  (2, -4);
            \draw[dashed, black, ultra thick] (2,-4) --  (3.5, -4);
            \draw[black, ultra thick] (2,-4) --  (2.75, -5.5);
            \draw[myorange, ultra thick] (-2,-4) --  (1, -2);
            \draw[dashed, myorange, ultra thick] (1, -2) --  (3.5, 0);
            \draw[myyellow, ultra thick] (3.5, 0) --  (2, -4);
    
            \node[above] at (0,0) {\tiny B};
            \node[left] at (-2,-4) {\tiny A};
            \node[below] at (2.3, -4) {\tiny C};
            \node[above] at (1.1, -1.9) {\tiny E};
            \node[below] at (3.5, -4) {\tiny F};
            \node[right] at (2.75, -5.5) {\tiny G};
            \node[above] at (3.5, 0) {\tiny D};
        \end{tikzpicture}
     \end{center}
\end{minipage}\hfill
\begin{minipage}[t]{0.57\textwidth}
    \myheading

    \vspace{-0.2cm}
    \begin{wrapfigure}[2]{l}{0.2\linewidth}
        \vspace{-0.6cm}
        \includegraphics[width=0.9\linewidth]{capital.png}
    \end{wrapfigure}
    
    \slshape ри продолжении стороны треугольника
    
    \hspace{0.1cm}\begin{tikzpicture}[scale=0.9]
        \draw[dashed, orange, ultra thick] (-0.5,-1) --  (0,0);
        \draw[myblue, ultra thick] (0,0) --  (0.25, -0.5);
        \draw[dashed, myblue, ultra thick] (0.25,-0.5) --  (0.5, -1);
        \draw[black, ultra thick] (-0.5,-1) --  (0.5, -1);
        \node[above] at (0,0) {\tiny B};
        \node[left] at (-0.5,-1) {\tiny A};
        \node[right] at (0.5, -1) {\tiny C};
    \end{tikzpicture}
    \raisebox{2em}{внешний угол}
    \begin{tikzpicture}
        \draw[black, ultra thick,fill=black] (2, -4) -- (1.7, -3.4)  arc(-250: -295: 0.7) -- cycle;
        \draw[black,ultra thick] (2, -4) -- (2.5, -4)  arc(-4:50:0.7) -- cycle;
        \node[below] at (2, -4) {\tiny C};
        \node[left] at (1.7, -3.4) {\tiny E};
        \node[right] at (2.5, -4) {\tiny F};
    \end{tikzpicture}
    \raisebox{2em}{будет больше}
    \raisebox{2em}{
        любого из противолежащих ему внутренних углов
    }
    \vfill
    \vspace{-0.7cm}
    \raisebox{0.3em}{\begin{tikzpicture}
        \draw[black,fill=black] (0, 0) -- (0.35, -0.65)  arc(-60:-115:0.7) -- cycle;
        \node[above] at (0,0) {\tiny B};
        \node[below] at (0.35, -0.65) {\tiny C};
        \node[below] at (-0.35, -0.65) {\tiny A};
    \end{tikzpicture}}
    \raisebox{2em}{или}
    \raisebox{0.3em}{\begin{tikzpicture}
        \draw[myblue,fill=myblue] (-2, -4) --  (-1.4,-4) arc(0:55:0.7) -- cycle;
        \node[above] at (-1.8, -3.5) {\tiny B};
        \node[below] at (-2, -4) {\tiny A};
        \node[below] at (-1.4,-4) {\tiny C};
    \end{tikzpicture}}
    \raisebox{2em}{.}
    \vspace{0cm}
    \upshape
    \begin{center}
    
    \vfill
        Сделаем
        \raisebox{0.2\height}{\begin{tikzpicture}
            \draw[myblue,ultra thick] (0,0) -- (1,0);
            \node[above] at (0,0) {\tiny B};
            \node[above] at (1,0) {\tiny E};
        \end{tikzpicture}}
        =
        \raisebox{0.2\height}{\begin{tikzpicture}[]
            \draw[dashed, myblue,ultra thick] (0,0) -- (1,0);
            \node[above] at (0,0) {\tiny E};
            \node[above] at (1,0) {\tiny C};
        \end{tikzpicture} (пр. I.{\tiny I0});}
    \vfill
    \vfill
        проведем
        \raisebox{0.2\height}{\begin{tikzpicture}
            \draw[myorange,ultra thick] (0,0) -- (1,0);
            \node[above] at (0,0) {\tiny A};
            \node[above] at (1,0) {\tiny E};
        \end{tikzpicture}}
        и продлим до
        \raisebox{0.2\height}{\begin{tikzpicture}
            \draw[dashed, myorange, ultra thick] (0,0) -- (1,0);
            \node[above] at (0,0) {\tiny E};
            \node[above] at (1,0) {\tiny D};
        \end{tikzpicture}}
         =
        \raisebox{0.2\height}{\begin{tikzpicture}
            \draw[myorange, ultra thick] (0,0) -- (1,0);
            \node[above] at (0,0) {\tiny A};
            \node[above] at (1,0) {\tiny E};
        \end{tikzpicture}}
    \vfill
    \vfill
        \vspace{0cm} % DO NOT REMOVE!!!
        \raisebox{1.8em}{проведем}
        \raisebox{1.8em}{\begin{tikzpicture}
            \draw[myyellow,ultra thick] (0,0) -- (1,0);
            \node[above] at (0,0) {\tiny C};
            \node[above] at (1,0) {\tiny D};
        \end{tikzpicture}}
        \raisebox{1.8em}{. В} 
        \begin{tikzpicture}
            \draw[dashed, myyellow, ultra thick] (-0.5,-1) --  (0,0);
            \draw[myblue, ultra thick] (0,0) --  (0.25, -0.5);
            \draw[myorange, ultra thick] (-0.5,-1) --  (0.25, -0.5);
            \node[above] at (0,0) {\tiny B};
            \node[left] at (-0.5,-1) {\tiny A};
            \node[right] at (0.25, -0.5) {\tiny E};
        \end{tikzpicture}
        \raisebox{1.8em}{ и }
        \begin{tikzpicture}
            \draw[dashed, myblue, ultra thick] (0.25,-0.5) --  (0.5, -1);
            \draw[dashed, myorange, ultra thick] (0.25, -0.5) --  (0.75, 0);
            \draw[myyellow, ultra thick] (0.75, 0) --  (0.5, -1);
            \node[below] at (0.5, -1) {\tiny C};
            \node[left] at (0.25, -0.5) {\tiny E};
            \node[above] at (0.75, 0) {\tiny D};
        \end{tikzpicture}
        \raisebox{1.8em}{;}
    \vfill
    \vfill
        \raisebox{1.8em}{\begin{tikzpicture}
            \draw[myblue ,ultra thick] (0,0) -- (1,0);
            \node[above] at (0,0) {\tiny B};
            \node[above] at (1,0) {\tiny E};
        \end{tikzpicture}}
        \raisebox{1.8em}{ = }
        \raisebox{1.8em}{\begin{tikzpicture}
            \draw[dashed, myblue,ultra thick] (0,0) -- (1,0);
            \node[above] at (0,0) {\tiny E};
            \node[above] at (1,0) {\tiny C};
        \end{tikzpicture}} 
        \raisebox{1.8em}{ , }
        \begin{tikzpicture}
            \draw[myyellow,fill=myyellow] (1, -2) -- (0.5, -2.3)  arc(-160:-235:0.7) -- cycle;
            \node[above] at (0.9, -1.6) {\tiny B};
            \node[below] at (0.4, -2.3) {\tiny A};
            \node[right] at (1, -2) {\tiny E};
        \end{tikzpicture}
        \raisebox{1.8em}{ = } 
        \raisebox{-0.4em}{\begin{tikzpicture}
            \draw[myyellow,fill=myyellow] (1, -2) -- (1.3, -2.6)  arc(-50:35:0.7) -- cycle;
            \node[below] at (1.4, -2.6) {\tiny C};
            \node[above] at (1.4, -1.7) {\tiny D};
            \node[left] at (1, -2) {\tiny E};
        \end{tikzpicture}}
    \vfill
    \vfill
        (пр. I.{\tiny 15}) и 
        \begin{tikzpicture}
            \draw[myorange ,ultra thick] (0,0) -- (1,0);
            \node[above] at (0,0) {\tiny A};
            \node[above] at (1,0) {\tiny E};
        \end{tikzpicture} =
        \begin{tikzpicture}
            \draw[dashed, myorange,ultra thick] (0,0) -- (1,0);
            \node[above] at (0,0) {\tiny E};
            \node[above] at (1,0) {\tiny D};
        \end{tikzpicture}  (постр.), 
    \vfill
    
    \vfill
        \raisebox{1.8em}{$\therefore$}
        \begin{tikzpicture}
            \draw[black,fill=black] (0, 0) -- (0.35, -0.65)  arc(-60:-115:0.7) -- cycle;
            \node[above] at (0,0) {\tiny B};
            \node[below] at (0.35, -0.65) {\tiny C};
            \node[below] at (-0.35, -0.65) {\tiny A};
        \end{tikzpicture}
        \raisebox{1.8em}{ = }
        \begin{tikzpicture}
            \draw[black, ultra thick,fill=black] (2, -4) -- (1.7, -3.4)  arc(-250: -295: 0.7) -- cycle;
            \node[above] at (2.3, -3.4) {\tiny D};
            \node[above] at (1.7, -3.4) {\tiny E};
            \node[below] at (2, -4) {\tiny C};
        \end{tikzpicture}
        \raisebox{1.8em}{(пр. I.{\tiny 4}),}
    \vfill
    \vfill
        \raisebox{1.8em}{$\therefore$}
        \begin{tikzpicture}
            \draw[black, ultra thick,fill=black] (2, -4) -- (1.7, -3.4)  arc(-250: -295: 0.7) -- cycle;
            \draw[black,ultra thick] (2, -4) -- (2.5, -4)  arc(-4:50:0.7) -- cycle;
            \node[below] at (2, -4) {\tiny C};
            \node[left] at (1.7, -3.4) {\tiny E};
            \node[right] at (2.5, -4) {\tiny F};
        \end{tikzpicture}
        \raisebox{1.8em}{ > }
        \begin{tikzpicture}
            \draw[black,fill=black] (0, 0) -- (0.35, -0.65)  arc(-60:-115:0.7) -- cycle;
            \node[above] at (0,0) {\tiny B};
            \node[below] at (0.35, -0.65) {\tiny C};
            \node[below] at (-0.35, -0.65) {\tiny A};
        \end{tikzpicture}
        \raisebox{1.8em}{.}
    \vfill
    Так же можно показать, что при \\
    \raisebox{1.5em}{продлении}
    \raisebox{1.5em}{\begin{tikzpicture}
        \draw[myblue, ultra thick] (0,0) --  (0.5, 0);
        \draw[dashed, myblue, ultra thick] (0.5, 0) --  (1, 0);
        \node[above] at (0,0) {\tiny B};
        \node[above] at (1,0) {\tiny C};
    \end{tikzpicture}}
    \raisebox{1.5em}{,} 
    \begin{tikzpicture}
        \draw[myorange,fill=myorange] (2, -4) -- (1.4, -4)  arc(185:290:0.7) -- cycle;
        \node[above] at (2, -4) {\tiny C};
        \node[below] at (2.3, -4.6) {\tiny G};
        \node[below] at (1.2,-3.8) {\tiny A};
    \end{tikzpicture}
    \raisebox{1.5em}{ > }
    \begin{tikzpicture}
        \draw[myblue,fill=myblue] (-2, -4) --  (-1.4,-4) arc(0:55:0.7) -- cycle;
        \node[above] at (-1.6, -3.5) {\tiny B};
        \node[below] at (-2, -4) {\tiny A};
        \node[below] at (-1.4,-4) {\tiny C};
    \end{tikzpicture} \\
    \raisebox{1.5em}{и, следовательно}
    \begin{tikzpicture}
        \draw[black, ultra thick,fill=black] (2, -4) -- (1.7, -3.4)  arc(-250: -295: 0.7) -- cycle;
        \draw[black,ultra thick] (2, -4) -- (2.5, -4)  arc(-4:50:0.7) -- cycle;
        \node[below] at (2, -4) {\tiny C};
        \node[left] at (1.7, -3.3) {\tiny E};
        \node[right] at (2.5, -4) {\tiny F};
    \end{tikzpicture}
    \raisebox{1.5em}{ который } \\
    \raisebox{1.3em}{ = }
    \begin{tikzpicture}
        \draw[myorange,fill=myorange] (2, -4) -- (1.4, -4)  arc(185:290:0.7) -- cycle;
        \node[above] at (2, -4) {\tiny C};
        \node[below] at (2.3, -4.6) {\tiny G};
        \node[below] at (1.2,-3.8) {\tiny A};
    \end{tikzpicture}
    \raisebox{1.3em}{будет > }
    \begin{tikzpicture}
        \draw[myblue,fill=myblue] (-2, -4) --  (-1.4,-4) arc(0:55:0.7) -- cycle;
        \node[above] at (-1.6, -3.5) {\tiny B};
        \node[below] at (-2, -4) {\tiny A};
        \node[below] at (-1.4,-4) {\tiny C};
    \end{tikzpicture}
    \raisebox{1.3em}{.} 
  
    \end{center}
    \begin{flushright}
    ч. т. д.
    \end{flushright}
\end{minipage}
\end{document}